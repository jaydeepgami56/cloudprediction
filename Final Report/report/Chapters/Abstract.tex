Infrastructure-As-A-Service (IAAS) is one of the emerging services of a Cloud Computing that provides virtual computing resources like storage device, network connection, hardware to the user on-demand in an elastic manner. But many challenges occur while using on-demand elastic services. The main challenge is a delay in resource initialization, when Any user tries to access the IAAS services it takes some time to initialize the service on a cloud. The delay of time very important while running business-related workloads on a cloud i.e web servers, storage server, mailing service. Prediction of the business-related workload provides better system utilization and optimal computing result. However, the challenging task to predict the workloads on a cloud since the loading of resources dramatically fluctuate in a short period. In Proposed model use Long-Short Term Memory(LSTM) to predict the CPU utilization on a cloud, it uses the GWA-T-12-Bitbrain \cite{shen2015statistical} dataset provides 1750 VM performance metrics. Comparative analysis of the proposed model with a time-series model provides a greater view of the performance of resource prediction.
  

\textbf{Keywords}:
Cloud Computing, LSTM, Time-Series, ARIMA, Neural Network, resource prediction


