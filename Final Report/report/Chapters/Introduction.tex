 
Cloud computing is one of the prominent computing technology that provides many services in a distributed environment. Based on the inclusion of extensive contextual foundation, Cloud Computing is classified into 3 services, Infrastructure-As-A-Service (IAAS), Platform-As-A-Service (PAAS), Software-As-A-Service (SAAS) \cite{chen2011overview}. The prominent feature of cloud computing is Virtualization. One of the big advantages of cloud computing is to provide all IT resources related to network, hardware, and storage on demand without any physical setup. Customize the system based on need and demand easily attach or detach the resource services. However, one of the flaws of this technology that when demand increase or too many requests to access the resources on a cloud then it takes some time to initialize the new resources which affects the performance and accessibility of the system. The purpose of predicting the resource usage in cloud computing to utilize the resources while managing workload between resources to minimize the execution time and improve the efficiency of the system on a cloud.

Over-provision of resources one of the main challenges in cloud computing. When any system is deployed in a cloud environment, it reserved some resources to manage the future workload. Reservation of resources helps to end-user to access the system during peak hours and improve the performance of the system but during the non-peak time, it is a wastage of resource which cost more to business vendors. If the resources are under-provisioned then there might be a chance of unavailability of services due to a spike in user requests. Lack of strategy and poor planning of scheduling resources during peak time may produce inefficiency in the performance of the system that directly affects the business that is running on a cloud environment \cite{yadav2016priority}.

 Achieving the high performance of a system running on a cloud is very complex. Proper resource allocation not only improves the efficiency of the system but also minimizes the cost of resource usage on a cloud. Therefore, any proposed model improving the efficiency of performance without any economic benefits not suitable in the environment. Optimal allocation of resources is one of the good strategies to improve performance. Scheduling of the resource is the key factor to minimize cost. There are important key factors consider while scheduling the resources i.e., demand of a resource, availability of existing resources, user requests workload on a system, scheduling algorithm. The main of scheduling to provide service to end-users with minimum response time with minimal cost. The proposed solution improves the efficiency of the performance which minimizes the cost of resource usage on a cloud \cite{alam2017reliability}.
 
 Armbrust et. al mentioned in their paper the inflexibility of cloud computing is to predict the resource utilization of systems \cite{duggan2017predicting}. One of the difficult tasks is to perform a resource prediction algorithm on the fluctuation of data, most of the computing tasks of any system perform on the virtual machine on a cloud. There are many key metrics i.e., CPU usage, memory usage, network throughput, CPU cores, network capacity, generated by a virtual machine that helps to predict the demand of workload and based on-demand scaling the resources. There is a traditional time-series model like Auto-Regressive Integrated Moving Average (ARIMA), Seasonal ARIMA (SARIMA) helps to forecast the resources based on demand\cite{zharikov2020adaptive}. But the traditional timer-series approach may not provide good results in multi-variant data systems or when there is the most significant amount of variation in historical data.
 
Neural Network is the key concept that solves the above problem, from past decades, Machine learning is very useful to solve the prediction problem in cloud computing\cite{duggan2017predicting,kumar2018workload}. There are many traditional machine learning algorithm provides measurable result but the most effective and accurate method is Neural Network (NN). The Neural Network model is the mimics of the brain neuron cells, based on several neurons passing information to the next layers that perform the complex calculations for prediction. CPU utilization is one of the important factors while a system running in the virtual machine on a cloud. The proposed model uses a Long-Short Term Memory (LSTM) recurrent neural network to predict the CPU usage, and comparative analysis of results with previous time-series models.
